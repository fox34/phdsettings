\documentclass[
    % Einseitiger oder zweiseitiger Druck?
    twoside=true,
    % Bezug: 12-Punkt Schriftgröße
    fontsize=11pt,
    % Randaufteilung, siehe Dokumentation "KOMA"-Script
    DIV=15,
    %DIV=calc % Alternativ automatisch berechnet
    % Bindekorrektur: Innen 17mm Platz lassen
    BCOR=17mm,
    % Trennlinien
    headsepline=true, footsepline=false,
    % Neue Kapitel im zweiseitigen Druck rechts beginnen lassen
    open=right,
    % Seitenformat A4
    a4paper,
    % Abstract einbinden
    %abstract=true,
    % Div. Verzeichnisse ins Inhaltsverzeichnis aufnehmen
    listof=totoc, bibliography=totoc,
    % Titelseite aktivieren
    %titlepage,
    % Kopf-/Fußzeile in die Satzspiegelberechnung mit einbeziehen
    headinclude=true, footinclude=false,
    % Europäischer Satz mit Abstand zwischen Absätzen
    parskip=half,
    % Gliederungsnummern ohne abschließenden Punkt darstellen
    numbers=noenddot,
    % Deutsche Rechtschreibung
    ngerman
% Dokumentenstil: "Buch" aus dem KOMA-Skript-Paket
]{scrbook}


% Autor und Eigenschaften, muss vor PhDSettings
\title{Beispieldokument}
\author{Steffen Manzer}
\date{\today}
\newcommand{\thesubject}{Handbuch}

% Lade PhDSettings
\usepackage{PhDSettings}

\usepackage[style=authoryear, sorting=nyt]{biblatex}
\addbibresource{Example.bib}

%%%%%%%%%%%%%%%%%%%%%%%%%
% Dokumentbeginn
%%%%%%%%%%%%%%%%%%%%%%%%%
\begin{document}

    %%%%%%%%%%%%%%%%%%%%%%%%%
    % Titelblatt, Vorwort, Verzeichnisse
    %%%%%%%%%%%%%%%%%%%%%%%%%

    % Hier z.B. Titelblatt
    % \input{...}
    
    % Beginn Nummerierung mit römischen Zahlen
    \pagenumbering{Roman}
    \setcounter{page}{1}
    
    % Hier z.B. Geleitworte
    % \input{...}
    
    % Inhaltsverzeichnis
    \tableofcontents
    \newpage

    % Abbildungsverzeichnis
    \listoffigures
    \newpage
    
    % Tabellenverzeichnis
    \listoftables
    \newpage
    
    
    
    %%%%%%%%%%%%%%%%%%%%%%%%%
    % Abkürzungen
    % Normalerweise in eigener Datei mit \input{...}
    %%%%%%%%%%%%%%%%%%%%%%%%%
    \chapter*{Abkürzungen}
    \addcontentsline{toc}{chapter}{Abkürzungen}

    \bgroup
    \renewcommand*{\arraystretch}{1.2} % Zeilenabstand in dieser Gruppe
    \begin{longtable}[l]{ A{l} Z{l} }
    
    % A
    \textbf{API}    & Application Programming Interface \\
    % B
    \vspace{1mm} \\
    \textbf{BaFin}  & Bundesanstalt für Finanzdienstleistungsaufsicht \\
    \textbf{BTC}    & Bitcoin \\
    % C
    \vspace{1mm} \\
    \textbf{CAD}    & Kanadischer Dollar \\
    \textbf{CNY}    & Chinesischer Yuan (Renminbi) \\
    % Usw.
    
    \end{longtable}
    \egroup
    \newpage
    
    
    
    %%%%%%%%%%%%%%%%%%%%%%%%%
    % Inhalt
    % Normalerweise in eigener Datei mit \input{...}
    %%%%%%%%%%%%%%%%%%%%%%%%%

    % Arabische Zahlen, von 1 an
    \pagenumbering{arabic}
    \setcounter{page}{1}

    \chapter{Beispieldokument für PhDSettings}
    
    \section{Abbildungen}
    \blindtext
    
    \begin{figure}[htb]
        \centering
        \vspace{3mm}
        \begin{tikzpicture}
            
            % Farben aus https://personal.sron.nl/~pault/
            \definecolor{PTOL_LIGHT_1}{HTML}{77AADD}
            \definecolor{PTOL_LIGHT_2}{HTML}{99DDFF}
            \definecolor{PTOL_LIGHT_3}{HTML}{44BB99}
            
            % Standardwerte für jeden Knoten in dieser Grafik
            \tikzset{every node/.style={font=\small\sffamily, align=center}}
            
            \node [circle, fill=PTOL_LIGHT_1, minimum width=60mm] (store_of_value) at (0, 0) {};
            \node [circle, fill=PTOL_LIGHT_2, minimum width=36mm, above left=9mm of store_of_value.center, anchor=center] (exchange_medium) {};
            \node [circle, fill=PTOL_LIGHT_3, minimum width=18mm, below left=7mm of exchange_medium.center, anchor=center] (unit_of_account) {};
            
            \node [xshift=-8mm, yshift=9mm] at (store_of_value.south east) {Wert-\\aufbewahrungs-\\funktion};
            \node [below=7mm of exchange_medium.north] {Tauschfunktion};
            \node at (unit_of_account) {Rechen-\\funktion};
            
        \end{tikzpicture}
    
        \caption{Hierarchie der Geldfunktionen}
        \source{Eigene Darstellung in Anlehnung an \cite{Ali2014EconomicsOfDigitalCurrencies}, S.~279.}
        \label{Grafik:Devisen_Geldfunktionen_Hierarchie}
    \end{figure}

    \blindtext
    
    
    
    \section{Tabellen}
    \blindtext
    
    \begin{table}[htb]\pushftn % \pushftn sammelt Fußnoten in Floats
    
        % Beschriftung
        \caption{Verschiedene Bitcoin-Stückelungen}
        
        % Erste Spalte: A = Ohne Rand links, Inhalt: Paragraph, 5cm
        % Zweite Spalte: R = Rechtsbündiger Paragraph, 5cm
        % Dritte Spalte: p = Linksbündiger Paragraph, 2cm
        % Letzte Spalte: Z = Ohne Rand rechts, Inhalt: X (autofill von tabularx)
        \begin{tabularx}{\textwidth}{ A{p{5cm}} R{3.8cm} p{2cm} Z{X} }
            
            % Rahmen
            \tablehead
            
            {Einheit} &
                {Menge [BTC]} &
                {Abkürzung} &
                {Anmerkung} \\
            
            % Abgrenzung Kopfzeile
            \tablebody
            
            Rechnerische Maximalmenge%
              \footnote{Berechnet gemäß $10^{-8} \cdot \sum_{i=0}^{32} (210\,000 \cdot \left \lfloor{50 \cdot 10^8 \cdot 2^{-i}}\right \rfloor )$, vgl. \cite{BitcoinWiki_ControlledSupply}.} &
                20.999.999,976\,900\,00 &
                &
                \\
            
            bitcoin &
                1,000\,000\,00 &
                BTC &
                Basiseinheit \\
            
            milli-bitcoin &
                0,001\,000\,00 &
                mBTC &
                SI-Präfix Milli: $10^{-3}$ \\
            
            micro-bitcoin, bit &
                0,000\,001\,00 &
                µBTC, Bit &
                SI-Präfix Mikro: $10^{-6}$ \\
            
            satoshi &
                0,000\,000\,01 &
                sat &
                Aktuell kleinste Einheit \\
            
            % Rahmen
            \tablefoot
        \end{tabularx}
        
        % Quelle und Label
        \source{Eigene Zusammenstellung, Stand April 2020.}
        \label{Tabelle:Krypto_Stueckelung}
    \end{table}\popftn
    % \popftn fügt gesammelte Fußnoten ein.
    % To Do: Position der eingefügten Fußnoten prüfen

    \blindtext

    
    %%%%%%%%%%%%%%%%%%%%%%%%%
    % Anhang
    % Hier nur Glossar.
    % Normalerweise in eigener Datei mit \input{...}
    %%%%%%%%%%%%%%%%%%%%%%%%%
    \chapter*{Glossar}
    \addcontentsline{toc}{chapter}{Glossar}
    \runinhead{Adresse} Bitcoin-Adressen sind eindeutige Zeichenketten, die die Funktion einer Kontonummer erfüllen.
        Eine Beispiel-Adresse sieht wie folgt aus: 1PC9""aZC4""hNX2""rmmr""t7uH""TfYA""S3hR""bph4UN.
    
    \runinhead{Bitcoin} Erste dezentrale, auf kryptografischen Funktionen basierende Währung, die ohne eine vertrauenswürdige Instanz funktioniert. [...]
    
    \runinhead{Blockchain} Sammlung beliebiger Daten bzw. Dokumente, die miteinander, typischerweise zeitlich, in Verbindung gebracht werden. [...]
    
    \runinhead{Node} Teilnehmer eines Peer-to-Peer-Netzwerkes.
    % ...
    
    
    
    %%%%%%%%%%%%%%%%%%%%%%%%%
    % Literatur
    %%%%%%%%%%%%%%%%%%%%%%%%%
    \printbibliography
    
\end{document}